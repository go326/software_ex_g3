
%% 本文
\begin{document}
\section{システム概要}\label{sec:sys}
本システムは,メインシステムとして,「管理者向けシステム」,「従業員向けシステム」,「アルバイト向けシステム」の3つを実装する.
「従業員向けシステム」には,「フロント担当」,「清掃担当」,「レストラン担当」のそれぞれの業務毎に,専用のサブシステムを実装する.
また,各システムに関連するサブシステムも併せて示す.
それぞれのサブシステムの詳細については,\ref{subsys}章で述べる.

\subsection{管理者向けシステム}
管理者向けシステムでは,本システムへのログイン後,ユーザ情報,マニュアル,質問情報の登録と管理を行うことができる.
また,本システムのログを参照することができる.
\begin{description}
\item[関連サブシステム] \\
ユーザ認証サブシステム,ユーザ情報管理サブシステム,ログ参照サブシステム,\\
質問情報管理サブシステム,マニュアル管理サブシステム
\end{description}

\subsection{従業員向けシステム}
従業員向けシステムでは,本システムへのログイン時,各業務毎に権限が与えられる.
全従業員の共通機能として,マニュアルと質問情報を閲覧することができる.

各業務の専用機能として,フロント担当は,予約情報の確認,管理,検索を行うことができる.
また,フロント担当は,清掃情報の管理と追加料金の管理も行うことができる.
清掃担当は,清掃状況の管理と確認を行うことができる.
レストラン担当は,追加料金の管理,食事情報の確認を行うことができる.
\begin{description}
\item[関連サブシステム] \\
ユーザ認証サブシステム,質問情報確認サブシステム,マニュアル確認サブシステム,\\
予約一覧表示サブシステム,予約情報検索サブシステム,予約管理サブシステム,\\
清掃情報管理サブシステム,清掃情報一覧表示サブシステム,追加料金管理サブシステム,\\
食事情報一覧表示サブシステム
\end{description}

\subsection{アルバイト向けシステム}
アルバイト向けシステムでは,本システムへのログイン後,マニュアルと質問情報の閲覧のみ行うことができる.
\begin{description}
\item[関連サブシステム] \\
ユーザ認証サブシステム,質問情報確認サブシステム,マニュアル確認サブシステム
\end{description}

\end{document}