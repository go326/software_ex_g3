\documentclass[main]{subfiles}

\begin{document}

\section{はじめに}
本システムは,ホテルUの問題点を解決するためのWEBシステムである.ホテルUの問題点は大きく分けて以下の通りである.

\begin{enumerate}
    \item お客様からのよくある質問に対しての回答を共有できていない

    \item マニュアルが有効活用されていない

    \item 情報の一括管理ができていない
\end{enumerate}

これらを解決するために本システムでは,新入社員やアルバイト向けにマニュアルの電子データ化やよくある質問と回答を閲覧できるようにする.また,紙媒体で行われている業務や情報を一括管理することにより,情報共有の漏れを防止しペーパーレス化を行う.これらによって,業務内容を効率化することに加え,サービスの向上をすることが本システムの目的である.


\section{システムの流れ}
本システムの利用者は,ホテルUの従事者である.その中でも,管理者,フロント,清掃,レストラン,アルバイトの業務ごとに分けてそれぞれ機能を実装する.

本システムを利用の一例とその過程を各利用者ごとに示す.すべての利用者は,配布した端末を使用し本システムに接続し,ユーザIDとパスワードを利用してログインする.

管理者向け機能では,本システムを利用するためのユーザ情報を登録する際のシステムの利用を示す.新入社員が本システムを利用するために,ユーザ情報を登録する.その際,ユーザ名とパスワードを新入社員が入力する.そして,新入社員が担当する業務の権限を管理者が付与する.これによりユーザ情報を登録し,登録されたユーザが本システムにログインし利用することができる.

フロント担当向け機能では,ホテルを予約しているお客様がチェックインする際のシステムの利用を示す.フロント担当が本システムの予約情報とお客様の予約情報を照合し,本システムの予約情報をチェックイン状態に変更する.

清掃担当向け機能では,部屋を清掃する際のシステムの利用を示す.清掃担当が本システムを利用し,清掃する部屋を確認し,清掃業務を行う.そして,清掃した部屋に対して清掃担当が,本システムの清掃情報を清掃済み状態に変更する.

レストラン担当向け機能では,当日のお客様が食事を追加で頼み追加料金が発生した際のシステムの利用を示す.レストランではお客様が食事を追加で頼んだ際に追加料金が発生する.その際,フロントで一括して料金を請求するために,本システムに追加料金を登録する.レストラン担当はお客様が食事を終了し,発生した追加料金を計算する.そして,追加料金の合計を本システムにレストラン担当が登録する.登録された追加料金を,フロント担当が確認してお客様に追加料金の請求を行う.


アルバイト向け機能では,マニュアルを確認する際のシステムの利用を示す.アルバイトが本システムのマニュアル情報一覧から,仕事内容のわからない仕事のマニュアルを選択し閲覧する.これにより,アルバイトはマニュアルを参考に仕事を行うことができる.さらに,アルバイト向け機能では,お客様から急に質問をされた場合,質問の回答例を確認することができる.マニュアルの閲覧と同じ方法で質問内容とその回答例を閲覧することができる.

\end{document}
