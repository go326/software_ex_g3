\section{データベース設計}

本章では,本システムで用いるリレーショナルデータベースの構造を示す.

\subsection{顧客情報テーブル}
まず,お客様の情報を格納するテーブルを表\ref{fig:data_cus}に示す.
\begin{table}
    \begin{center}
    \begin{tabular}{|c|c|c|c|}
        \hline
        属性       & データ型      &  制約条件     & 備考 \\ \hline \hline
        予約ID     & INTEGER       & PRIMARY KEY  &  - \\ \hline
        宿泊日     & DATE          &   NOT NULL   &   (yyyy/mm/dd) \\ \hline
        予約日     & DATE          &   NOT NULL   &   (yyyy/mm/dd) \\ \hline
        宿泊数     & INTEGER       &   NOT NULL    &  - \\ \hline 
        氏名       & CHAR(16)      &   NOT NULL    &  -  \\\hline
        都道府県   & CHAR(5)       &   NOT NULL     & - \\ \hline
        住所       & CHAR(128)     & NOT NULL      & - \\  \hline
        電話番号   & INTEGER       & NOT NULL       & - \\   \hline
        大人       & INTEGER       & NOT NULL       & - \\   \hline
        子供       & INTEGER       & -             & - \\ \hline      
        プラン     & CHAR(32)      & NOT NULL      & - \\  \hline 
        夕食の有無 & CHAR(2)       & NOT NULL       & -  \\ \hline
        夕食メニュー   & CHAR(32)      & -             & -  \\ \hline
        朝食の有無 & CHAR(2)       & NOT NULL       & -  \\ \hline
        朝食メニュー   & CHAR(32)      & -             & -  \\ \hline
        部屋番号1  & CHAR(4)       & FOREIGN KEY   & references room \\ \hline
        部屋番号2  & CHAR(4)      & FOREIGN KEY   & references room \\ \hline
        部屋番号3  & CAHR(4)       & FOREIGN KEY   & references room \\ \hline
        チェックイン & CHAR(4)    & -             & -  \\ \hline
        備考 & CHAR(1024)    & -             & -  \\ \hline
        \end{tabular}
        \label{fig:data_cus} 
        \caption{customer}
    \end{center}
    \end{table}

    \subsection{ルーム情報テーブル}
まず,お部屋の情報を格納するテーブルを表\ref{fig:data_room}に示す.

 \begin{table}
\begin{center}
\begin{tabular}{|c|c|c|c|}
\hline
    属性       & データ型      &  制約条件     & 備考 \\ \hline \hline
    部屋番号     & CHAR(4)       & PRIMARY KEY  &  - \\ \hline
    掃除状況     & CHAR(2)          &   -          &   - \\ \hline
    \end{tabular}
    \label{fig:data_room} 
    \caption{room}
\end{center}
\end{table}
 
\subsection{社員情報テーブル}
まず,従業員の情報を格納するテーブルを表\ref{fig:data_emp}に示す.
\begin{table}
\begin{center}
\begin{tabular}{|c|c|c|c|}
\hline
    属性       & データ型      &  制約条件     & 備考 \\ \hline \hline
    社員番号   & CHAR(4)       & PRIMARY KEY  &  - \\ \hline
    ユーザ名     & CHAR(16)         &   -          &   - \\ \hline
    漢字氏名     & CHAR(8)       &   -          &   - \\ \hline
    パスワード     & INTEGER       & -            &  ハッシュ化したもの \\ \hline 
    \end{tabular}
    \label{fig:data_emp} 
    \caption{employees}
\end{center}
\end{table}

\subsection{ログ情報テーブル}
まず,ログの情報を格納するテーブルを表\ref{fig:data_log}に示す.

\begin{table}
\begin{center}
    \begin{tabular}{|c|c|c|c|}
    \hline
    属性       & データ型      &  制約条件     & 備考 \\ \hline \hline
    日時       & DATE          & PRIMARY KEY  &  (YYYY/MM/DD hh:mm:ss) \\ \hline
    変更者     & CHAR(4)       & PRIMARY KEY, FOREING KEY &  REFERENCES employees \\ \hline
    変更業務    & CHAR(16)     &   -          &   - \\ \hline
    変更テーブル    & CHAR(8)      & -            &  - \\ \hline 
    変更行       & CHAR(16)      & -            &  主キーで管理  \\\hline
    変更属性   & CHAR(5)       & -             &  - \\ \hline
    変更前       & CHAR(128)     & -             & - \\  \hline
    変更後   & INTEGER       & -             & - \\   \hline
    \end{tabular}
\label{fig:data_log} 
\caption{log}
\end{center}
\end{table}

